\section{Proposed Architecture} \label{arch}

\subsection{Overview}

The TorCoin architecture implements the novel TorCoin and TorPath protocols. We
describe both in later sections, but in brief, the TorCoin protocol mines
coins, and the TorPath protocol assigns a circuit (entry, middle, and exit
servers) to each client.

TorCoin runs as a standalone service, and requires very little modification of
the core Tor codebase. Tor clients and relays operate as usual, but receive
circuit assignments from \textit{assignment servers} instead of directory 
servers. Separately, a \textit{TorCoin Miner} on each machine mines TorCoins
by monitoring the throughput of the local Tor TLS tunnel, and communicating with its circuit neighbors via the TorCoin algorithm.

Figure 3.1 shows a basic overview of this architecture.

\begin{figure}
  \centering
    \includegraphics[scale=0.3]{architecture.pdf}
  \caption{High level TorCoin system architecture for clients and relays. A \textit{TorPath Client} assigns Tor circuits to clients via the TorPath protocol, described in the next section. A \textit{TorCoin Miner} mines TorCoins and stores them in a \textit{TorCoin Wallet}. Each \textit{Tor Client} and \textit{Tor Relay} operates as usual, but on circuits assigned via the TorPath protocol.}
\end{figure}


% TorPath is an anonymous cooperative routing protocol that randomly assigns
% circuits to Tor clients using decentralized, cryptographically verifiable
% methods. It consists of groups of assignment servers, which are ``trusted'' in
% the same way as Tor directory servers, and a TorPath client to communicate
% with them.  Groups of assignment servers use the distributed TorPath protocol
% to  keep track of available relays and assign circuits to clients.

% The TorCoin algorithm is used to reward nodes for transferring bandwidth. It
% consists  of a TorCoin miner and a TorCoin Wallet. The TorCoin miner measures
% Tor bandwidth by monitoring the  throughput of the local Tor TLS tunnel,
% allowing us to avoid modifying internal  Tor code. It also sends out TorCoin
% packets that serve as ``proof-of-bandwidth''. The TorCoin Wallet is a
% cryptographic wallet for storage of coins and transactions. When the miner
% discovers a new TorCoin, it adds it to the blockchain with all the information
% necessary for anyone to verify it.


% Original text:
% \subsection{TorPath Components} TorPath is an anonymous cooperative routing
% scheme that randomly assigns circuits to Tor clients using decentralized,
% cryptographically verifiable methods. It requires groups of assignment servers,
% which are ``trusted'' in the same way as Tor directory servers, and a TorPath
% client to communicate with them. 

% \subsubsection{Assignment Server} A small set of trusted nodes run the
% Assignment Servers, which perform similar roles to the current Tor directory 
% servers. Groups of assignment servers use the distributed TorPath protocol to 
% keep track of available relays and assign circuits to clients.

% \subsubsection{TorPath Software} Clients and relays install the TorPath Software to
% communicate with Assignment Servers, using the TorPath protocol to retrieve
% circuits (clients). 
% % To avoid modifying Tor client code, TorPath can use a local 
% % DNS proxy to redirect requests to directory servers to the new assignment servers.

% \subsection{TorCoin Components} These components implement the TorCoin algorithm
% in order to reward nodes for transferring bandwidth. We describe the TorCoin 
% algorithm in depth in its own section, so here we only briefly outline the 
% components required to implement it.

% \subsubsection{TorCoin Miner} To mine TorCoins, clients and relays that are part
% of a circuit communicate via the TorCoin miner, which implements the TorCoin
% algorithm. The TorCoin miner measures Tor bandwidth by monitoring the the 
% throughput of the local Tor TLS tunnel, allowing us to avoid modifying internal 
% Tor code.

% \subsubsection{TorCoin Wallet} Since TorCoin is based on the BitCoin protocol,
% it uses a cryptographic wallet for storage of coins and transactions. When the
% miner discovers a new TorCoin, it adds it to the blockchain with all the
% information necessary fro anyone to verify it.