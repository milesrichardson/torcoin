\section{Proposed Architecture} \label{arch}

\subsection{Overview}

TorCoin runs as a standalone service, and requires no modification of the core
Tor codebase. The system can be broken into four components, which share code
across clients and relays, but behave differently depending on role. Figure 3.1
shows a basic overview of their architecture.

\begin{figure}
  \centering
    \includegraphics[scale=0.3]{architecture.pdf}
  \caption{High level TorCoin system architecture. Dotted lines communicate via
  the TorPath protocol, solid lines TorCoin, and dashed lines native Tor.Note
  that a ``trusted node''replaces a directory server, and actually improves on
  its anonymity properties.}
\end{figure}

We will briefly describe the role of each component in the system. Then, the
next two sections will describe the detailed implementations of the TorPath
protocol and TorCoin algorithm.

\subsection{TorPath Components} TorPath is an anonymous cooperative routing
scheme that randomly assigns circuits to Tor clients using decentralized,
cryptographically verifiable methods. It requires groups of assignment servers,
which are ``trusted'' in the same way as Tor directory servers, and a TorPath
client to communicate with them. In the TorPath section, we will show that
TorPath does not reduce the anonymity  provided by Tor, but in fact increases it.

\subsubsection{Assignment Server} A small set of trusted nodes run the
Assignment Server, which overrides the current Tor directory servers. Groups of
assignment servers use the distributed TorPath protocol to assign circuits to
clients and distribute access control lists (ACLs) to relays.

\subsubsection{TorPath Client} Clients and relays install the TorPath Client to
communicate with Assignment Servers, using the TorPath protocol to retrieve
circuits (clients) or ACLs (relays). To avoid modifying Tor client code, TorPath
can use a local DNS proxy to redirect requests to directory servers to the new
assignment servers.

\subsection{TorCoin Components} These components implement the TorCoin algorithm
in order to reward nodes for transferring bandwidth. We describe the TorCoin algorithm in depth in its own section, so here we only briefly outline the 
components required to implement it.

\subsubsection{TorCoin Miner} To mine TorCoins, clients and relays that are part
of a circuit communicate via the TorCoin miner, which implements the TorCoin
algorithm. The TorCoin miner allows us to avoid modifying internal Tor code,
because it measures Tor bandwidth by monitoring the the throughput of the local
Tor TLS tunnel.

\subsubsection{TorCoin Wallet} Since TorCoin is based on the BitCoin protocol,
it uses a cryptographic wallet for storage of coins and transactions. When the
miner discovers a new TorCoin, it adds it to the blockchain with all the
information necessary fro anyone to verify it.