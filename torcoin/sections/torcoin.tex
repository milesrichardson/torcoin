\subsection{TorCoin Mining}

Mining TorCoin requires transferring bandwidth. We introduce a novel method of 
\textit{onion hashing} to prove end-to-end bandwidth transfer across a Tor circuit.

\begin{figure}[H]
  \centering
    \includegraphics[scale=0.3]{torcoin_cycle.pdf}
  \caption{TorCoin Onion Hashing. Each circuit member has a generator and a 
  verifier of hash wrappers. Here we represent private keys as distinct shapes.}
\end{figure}

\subsubsection{Onion Hashing Algorithm}
\begin{enumerate}
\item Every $m$ Tor packets, client sends a ``TorCoin packet'' containing a hash
attempt $h_0$ likely to generate a TorCoin.
\item Relays encrypt the TorCoin packet using their own public keys for that route, 
and send it to the next relay in the circuit.
\item The exit relay adds its own key and sends it back to the middle relay 
alongwith its own key.
\item Each relay then verifies the final hash and sends it to the previous relay 
in the circuit, along with its own key.
\item The client verifies the correctness of the hash and the keys, thus proving
bandwidth transfer.
\end{enumerate}

Formally:

\begin{verbatim}
Client sends to A: T0 (its hash attempt)
A sends to B     : Hash(T0 + K1) = Ta
B sends to C     : Hash(Ta + K2) = Tb
C computes       : Hash(Tb + K3) = Tc
C sends to B     : (Tc, K3) to verify.
B sends to A     : (Tc, K3, Tb, K2) to verify.
A sends to client: (Tc, K3, Tb, K2, Ta, K1) to verify.
\end{verbatim}
where $Ki$ are the public keys of each relay.

\subsubsection{TorCoin}
We combine onion hashing with the BitCoin protocol to achieve TorCoin. 
To mine a TorCoin, the exit relay's final TorCoin packet should have a requisite
number of zeros in its lower order bits. If this is satisfied, the exit relay's
claim is verified by each of the relays in the chain.
If the claim is verified, the client's TorCoin Wallet writes it to the blockchain,
along with the following information:

\begin{itemize}
\item Timestamp of consensus group.
\item Client's Private key.
\item Circuit signature.
\item TorCoin hash.
\end{itemize}

All the information necessary for verifying proof-of-bandwidth is in the blockchain. 
Any interested party can then verify that the route signature is authentic 

\subsection{Security Considerations}

\subsubsection{Random Group Selection}
The TorPath protocol increases robustness of TorCoin to attackers. Its random group selection system inhibits attackers from deterministically placing themselves in a group.
We assume that atleast one of the assignment servers in the group is honest. In this case, the group as a whole is able to retain privacy and anonymity
We could make the system even more secure by randomizing group assignment, instead of just taking temporal locality to be the only criterion.

\subsubsection{Colluding to Form Circuits}
All honest relays and clients enforce the packet limit of $m$. If TorCoin packets 
are sent more frequently, they refuse to forward it to the next relay. If they are
sent less frequently, the relays are reported to the assignment server. The
assignment servers can keep track of relays that are frequently reported.

Thus, the attacker needs to control all four components of a route to mint a TorCoin fraudulently. If an adversary controls up to half the network, there is a probability of only $(\frac{1}{2})^4 = \frac{1}{16}$ that an adversary client gets a path of three colluding relays. In practice, gaining control of half of the entire Tor client and relay network is practically impossible. 

\subsection{Drawbacks}
The TorPath network is not backwards compatible with the existing Tor network, due to the fundamental differences of route assignment and access control, which are missing in Tor, but are necessary for the TorPath and TorCoin schemes to work.

However, our scheme does not preclude a given physical relay or server from running both services at the same time. They will, of course, only get paid for the TorCoin traffic. In time, we expect that a majority of the current Tor relay operators will switch over to the TorPath network, since TorPath has the same security guarantees as Tor provides, while also recompensing the relay operators for their costs.

Backwards-compatible Tor incentivization schemes like Tortoise\cite{acsac11-tortoise}, Gold Star\cite{incentives-fc10}, LIRA and Opportunistic Bandwidth measurement have all had significant drawbacks like the usage of Eigenspeed for bandwidth measurement or the establishment of a central bank to keep track of their tokens. These all introduce significant infrastructure into the Tor network and sometimes partition the anonymity sets